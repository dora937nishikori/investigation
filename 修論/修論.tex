\documentclass{ltjarticle}
\usepackage{geometry}
% ページの余白を1.25インチにする
\geometry{
    left=1.25truein,
    right=1.25truein,
    top=1.25truein,
    bottom=1.25truein,
}

\begin{document}
\begin{titlepage}
    \begin{center}
        {\Large 令和5年}
        \vspace{10truept}

        {\Large 修士論文草稿}
        \vspace*{180truept}

        {\Huge タイトル} 
        \vspace{160truept}

        {\Large 指導教員:相馬 隆郎}
        \vspace{30truept}

        {\Large 東京都立大学大学院}
        \vspace{10truept}

        {\Large 電子情報システム工学域}
        \vspace{30truept}

        {\Large 学修番号:22861651}
        \vspace{10truept}
        
        {\Large 氏名:西原涼介}
    \end{center}
\end{titlepage}
\noindent
{\LARGE 論文要旨}
\vspace{20truept}

ここから論文要旨

% 目次
\newpage
\tableofcontents
\clearpage

\part{はじめに}
\section{研究背景}
近年,Amazon や楽天市場などの大手 EC サイト以外に
も多くの EC サイトが普及し始め,その利用者も急増して
いる.そして商品を購入する際に EC サイトのレビューを
参考にしている利用者の割合は約 70%と言われていて,そ
の中でもレビューの信頼性を重要視している人が多いこと
が明らかになっている.また,企業にとっても EC サイトの
レビューからユーザーの嗜好や意見を分析し,マーケティ
ングに活用することが重要な課題となっている.そのため,
EC サイトのレビューの信頼性を評価する評判分析や口コ
ミ分析に関する研究が多く行われている.EC サイトのレビ
ューの他にも,近年では YouTube のような動画投稿サイト
や SNS での自社製品・サービスの宣伝を行う企業が増えて
きている.そして,その宣伝に対するユーザーのコメント
も,他のユーザーが商品の購入を検討する際の重要な判断
材料になり得ると考えられる.つまり,SNS や YouTube 上
での商品の宣伝に対するコメントは,EC サイトのレビュー
と同等の機能を持ち,評判分析の対象になると考えられる.
しかし,SNS や YouTube はその特性上,誰でも気軽にコメ
ントを投稿できるため,商品やサービスに関係ないコメン
トが多数存在する.そこで,本研究では Biterm Topic Model
を用いた商品に関するトピック抽出によって,YouTube 上
で自社製品やサービスを宣伝している動画に対するユーザ
ーのコメントから,その商品やサービスに対して関連性が
高いコメントを抽出するシステムを作成した.

\section{関連研究}
\subsection{BTMの解釈性の向上}
先行研究の始まり

\end{document}
\documentclass{ltjarticle}
\usepackage{geometry}
% ページの余白を1.25インチにする
\geometry{
    left=1.25truein,
    right=1.25truein,
    top=1.25truein,
    bottom=1.25truein,
}

\begin{document}
\begin{titlepage}
    \begin{center}
        {\Large 令和5年}
        \vspace{10truept}

        {\Large 修士論文草稿}
        \vspace*{180truept}

        {\Huge タイトル} 
        \vspace{160truept}

        {\Large 指導教員:相馬 隆郎}
        \vspace{30truept}

        {\Large 東京都立大学大学院}
        \vspace{10truept}

        {\Large 電子情報システム工学域}
        \vspace{30truept}

        {\Large 学修番号:22861651}
        \vspace{10truept}
        
        {\Large 氏名:西原涼介}
    \end{center}
\end{titlepage}
\noindent
{\LARGE 論文要旨}
\vspace{20truept}

ここから論文要旨

% 目次
\newpage
\tableofcontents
\clearpage

\part{はじめに}
\vspace{10truept}
\section{研究背景}
近年, Amazonや楽天市場などの大手ECサイトをはじめ、数多くのECサイトが普及し,  その利用者も急増して
いる. そして, 商品を購入する際にECサイトのレビューを参考にしている利用者の割合は約70%と言われていて, そ
の中でもレビューの信頼性を重要視している人が多いことが明らかになっている. また, 多くの企業にとって, ECサイトの
レビューからユーザーの嗜好や意見を分析し,マーケティングに活用することが重要な課題となっている.そのため,
ECサイトのレビューの信頼性や参考になるかどうかを評価する評判分析や口コミ分析, レビューを様々なトピックに分類する文書分類に関する
研究が多く行われている. 例えば, 関連研究の項で詳しく紹介する「機械学習を用いた自然言語処理による商品レビューの評価」[1]では, Amazonの商品レビュー
を機械学習を用いて参考になる順に並びかえるシステムの構築, 及びその評価に関する研究を行っている。
また近年では, 従来のECサイトや商品のWebページ以外にも, YouTubeのような
動画投稿サイトやX(旧Twitter)やInstagramなどのSNSで自社製品・サービスの宣伝を行う企業が増えてきている. 
それにつれて, 商品を購入する際にSNSやYouTube上でその商品を宣伝している投稿を参考にしている人も増加している. 
そのため, SNSやYouTube上の広告に対するユーザーのコメントも, 他のユーザーが商品の購入を検討する際の重要な判断
材料になり得ると考えられる. つまり, SNSやYouTube上での商品の宣伝に対するコメントは, ECサイトのレビュー
と同等の機能を持ち, その信頼性や参考になるかどうかが重要になるため, 
評判分析や文書分類の研究の対象になると考えられる. ここで, SNSやYouTubeは商品レビューのページとは異なり, 
誰でも気軽にコメントを投稿できたり, その投稿内容も自由という特性上, 商品やサービスに関係ないコメントが
多数存在する.

そこで, 本研究では分析対象をYouTube上で自社製品やサービスを宣伝している動画に対するユーザーのコメントとし, 
トピックモデルの一種であるBiterm Topic Modelによる商品に関するトピック抽出を用いて, 
その動画に対するユーザーのコメントから, 宣伝している商品やサービスに対して関連性が高いコメントを
抽出するシステムの作成, 及び作成したシステムの人手に対する精度の検証を行った. 

本論文の第Ⅰ部では, ECサイトのレビューにおける評判分析やトピックモデルを用いた文書分類に関する関連研究の紹介, 
また本研究の研究目的を明確に説明する.
第Ⅱ部では, 本研究で用いる二つのトピックモデルの説明, 及び提案手法のシステムや実装方法について説明する.
第Ⅲ部では, 実際のYouTube上の動画に対するコメントを用いた実験結果を述べる.
第Ⅳ部では, 実験結果をもとに考察した提案手法の有効性や将来性について述べる. 

\newpage
\section{関連研究}
本研究を進めるにあたり, 研究テーマの方向性決めや研究課題の発見, 及び本研究で用いている技術に関して参考にした
論文を4つ紹介する. 
\subsection{機械学習を用いた自然言語処理による商品レビューの評価[1]}
この研究では, ユーザーが商品レビューを読んで参考になったかどうかを評価する機能が備わっていない
ECサイトの場合に, 数多くあるレビューから参考になる情報を探す必要がある問題に着目し, 
機械学習を用いた自然言語処理の手法で分析, 評価を行い, レビューを参考になる順番に並び替える
システムの構築を目的としている. そして並び替えた順番が正しいかどうかを評価するために, クイックソートを利用した
新しい評価法であるQE法を提案している. 
この研究のシステム図が図1である.





\subsection{関連研究2}
関連研究2の始まり

\section{研究目的}
研究目的

\newpage
\part{提案手法}
\section{トピックモデル}
トピックモデルの説明
\subsection{Latent Dirichlet Allocation}
LDAの説明
\subsection{Biterm Topic Model}
BTMの説明
\section{提案システム}
提案システムの概要、図
\subsection{データ収集}
\subsection{前処理手法}

\newpage
\part{実験結果}
\section{実験目的、仮説}

\end{document}